\documentclass[a4paper,11pt]{article}

% Korean language support
\usepackage{kotex}

% Fullpage layout (minimal margins)
\usepackage{fullpage}

% Minted package for code highlighting
\usepackage{minted}

% Other useful packages
\usepackage{graphicx}
\usepackage{hyperref}
\usepackage{amsmath}
\usepackage{amssymb}

% Document metadata
\title{Can LLMs Judge Parser Error Clarity? A Minimal Viable Experiment with Happy and Megaparsec}
% 다른 제목 후보 예시:
% LLM-as-a-Judge for Parser Error Clarity: A Controlled Validation with LR and LL Parsing Strategies
\author{저자명}
\date{\today}

\begin{document}

\maketitle

\begin{abstract}
여기에 초록을 작성하세요.
\end{abstract}

\section{서론}
여기에 서론을 작성하세요.

LLM을 심판으로 파서 오류의 명확성 평가 방법에 대한 연구입니다.

\section{본론}

여기에 본론을 작성하세요.

\subsection{코드 예제}

minted 패키지를 사용한 코드 하이라이팅 예제입니다:

\begin{minted}{python}
def hello_world():
    print("안녕하세요, 세계!")
    
if __name__ == "__main__":
    hello_world()
\end{minted}

\section{결론}

여기에 결론을 작성하세요.

\end{document}
